\documentclass{article}
\usepackage{ifthen}
\usepackage{amssymb}
\usepackage{multicol}
\usepackage{graphicx}
\usepackage[absolute]{textpos}

%\usepackage{xspace,rotating,calligra,dsfont,ifthen}
\usepackage{xspace,rotating,dsfont,ifthen}
\usepackage[spanish,activeacute]{babel}
\usepackage[utf8]{inputenc}
\usepackage{pgfpages}
\usepackage{pgf,pgfarrows,pgfnodes,pgfautomata,pgfheaps,xspace,dsfont}
\usepackage{listings}
\usepackage{multicol}


\makeatletter

\@ifclassloaded{beamer}{%
  \newcommand{\tocarEspacios}{%
    \addtolength{\leftskip}{4em}%
    \addtolength{\parindent}{-3em}%
  }%
}
{%
  \usepackage[top=1cm,bottom=2cm,left=1cm,right=1cm]{geometry}%
  \usepackage{color}%
  \newcommand{\tocarEspacios}{%
    \addtolength{\leftskip}{5em}%
    \addtolength{\parindent}{-3em}%
  }%
}

\newcommand{\encabezadoDeProc}[4]{%
  % Ponemos la palabrita problema en tt
%  \noindent%
  {\normalfont\bfseries\ttfamily proc}%
  % Ponemos el nombre del problema
  \ %
  {\normalfont\ttfamily #2}%
  \
  % Ponemos los parametros
  (#3)%
  \ifthenelse{\equal{#4}{}}{}{%
  \ =\ %
  % Ponemos el nombre del resultado
  {\normalfont\ttfamily #1}%
  % Por ultimo, va el tipo del resultado
  \ : #4}
}

\newcommand{\encabezadoDeTipo}[2]{%
  % Ponemos la palabrita tipo en tt
  {\normalfont\bfseries\ttfamily tipo}%
  % Ponemos el nombre del tipo
  \ %
  {\normalfont\ttfamily #2}%
  \ifthenelse{\equal{#1}{}}{}{$\langle$#1$\rangle$}
}

% Primero definiciones de cosas al estilo title, author, date

\def\materia#1{\gdef\@materia{#1}}
\def\@materia{No especifi\'o la materia}
\def\lamateria{\@materia}

\def\cuatrimestre#1{\gdef\@cuatrimestre{#1}}
\def\@cuatrimestre{No especifi\'o el cuatrimestre}
\def\elcuatrimestre{\@cuatrimestre}

\def\anio#1{\gdef\@anio{#1}}
\def\@anio{No especifi\'o el anio}
\def\elanio{\@anio}

\def\fecha#1{\gdef\@fecha{#1}}
\def\@fecha{\today}
\def\lafecha{\@fecha}

\def\nombre#1{\gdef\@nombre{#1}}
\def\@nombre{No especific'o el nombre}
\def\elnombre{\@nombre}

\def\practicas#1{\gdef\@practica{#1}}
\def\@practica{No especifi\'o el n\'umero de pr\'actica}
\def\lapractica{\@practica}


% Esta macro convierte el numero de cuatrimestre a palabras
\newcommand{\cuatrimestreLindo}{
  \ifthenelse{\equal{\elcuatrimestre}{1}}
  {Primer cuatrimestre}
  {\ifthenelse{\equal{\elcuatrimestre}{2}}
  {Segundo cuatrimestre}
  {Verano}}
}


\newcommand{\depto}{{UBA -- Facultad de Ciencias Exactas y Naturales --
      Departamento de Computaci\'on}}

\newcommand{\titulopractica}{
  \centerline{\depto}
  \vspace{1ex}
  \centerline{{\Large\lamateria}}
  \vspace{0.5ex}
  \centerline{\cuatrimestreLindo de \elanio}
  \vspace{2ex}
  \centerline{{\huge Pr\'actica \lapractica -- \elnombre}}
  \vspace{5ex}
  \arreglarincisos
  \newcounter{ejercicio}
  \newenvironment{ejercicio}{\stepcounter{ejercicio}\textbf{Ejercicio
      \theejercicio}%
    \renewcommand\@currentlabel{\theejercicio}%
  }{\vspace{0.2cm}}
}


\newcommand{\titulotp}{
  \centerline{\depto}
  \vspace{1ex}
  \centerline{{\Large\lamateria}}
  \vspace{0.5ex}
  \centerline{\cuatrimestreLindo de \elanio}
  \vspace{0.5ex}
  \centerline{\lafecha}
  \vspace{2ex}
  \centerline{{\huge\elnombre}}
  \vspace{5ex}
}


%practicas
\newcommand{\practica}[2]{%
    \title{Pr\'actica #1 \\ #2}
    \author{Algoritmos y Estructuras de Datos I}
    \date{Primer Cuatrimestre 2018}

    \maketitlepractica{#1}{#2}
}

\newcommand \maketitlepractica[2] {%
\begin{center}
\begin{tabular}{r cr}
 \begin{tabular}{c}
{\large\bf\textsf{\ Algoritmos y Estructuras de Datos I\ }}\\
Primer Cuatrimestre 2018\\
\title{\normalsize Gu\'ia Pr\'actica #1 \\ \textbf{#2}}\\
\@title
\end{tabular} &
\begin{tabular}{@{} p{1.6cm} @{}}
\includegraphics[width=1.6cm]{logodpt.jpg}
\end{tabular} &
\begin{tabular}{l @{}}
 \emph{Departamento de Computaci\'on} \\
 \emph{Facultad de Ciencias Exactas y Naturales} \\
 \emph{Universidad de Buenos Aires} \\
\end{tabular}
\end{tabular}
\end{center}

\bigskip
}


% Simbolos varios

\newcommand{\ent}{\ensuremath{\mathds{Z}}}
\newcommand{\float}{\ensuremath{\mathds{R}}}
\newcommand{\bool}{\ensuremath{\mathsf{Bool}}}
\newcommand{\True}{\ensuremath{\mathrm{true}}}
\newcommand{\False}{\ensuremath{\mathrm{false}}}
\newcommand{\Then}{\ensuremath{\rightarrow}}
\newcommand{\Iff}{\ensuremath{\leftrightarrow}}
\newcommand{\implica}{\ensuremath{\longrightarrow}}
\newcommand{\IfThenElse}[3]{\ensuremath{\mathsf{if}\ #1\ \mathsf{then}\ #2\ \mathsf{else}\ #3\ \mathsf{fi}}}
\newcommand{\In}{\textsf{in }}
\newcommand{\Out}{\textsf{out }}
\newcommand{\Inout}{\textsf{inout }}
\newcommand{\yLuego}{\land _L}
\newcommand{\oLuego}{\lor _L}
\newcommand{\implicaLuego}{\implica _L}
\newcommand{\existe}[3]{\ensuremath{(\exists #1:\ent) \ #2 \leq #1 < #3 \ }}
\newcommand{\paraTodo}[3]{\ensuremath{(\forall #1:\ent) \ #2 \leq #1 < #3 \ }}

% Símbolo para marcar los ejercicios importantes (estrellita)
\newcommand\importante{\raisebox{0.5pt}{\ensuremath{\bigstar}}}


\newcommand{\rango}[2]{[#1\twodots#2]}
\newcommand{\comp}[2]{[\,#1\,|\,#2\,]}

\newcommand{\rangoac}[2]{(#1\twodots#2]}
\newcommand{\rangoca}[2]{[#1\twodots#2)}
\newcommand{\rangoaa}[2]{(#1\twodots#2)}

%ejercicios
\newtheorem{exercise}{Ejercicio}
\newenvironment{ejercicio}[1][]{\begin{exercise}#1\rm}{\end{exercise} \vspace{0.2cm}}
\newenvironment{items}{\begin{enumerate}[a)]}{\end{enumerate}}
\newenvironment{subitems}{\begin{enumerate}[i)]}{\end{enumerate}}
\newcommand{\sugerencia}[1]{\noindent \textbf{Sugerencia:} #1}

\lstnewenvironment{code}{
    \lstset{% general command to set parameter(s)
        language=C++, basicstyle=\small\ttfamily, keywordstyle=\slshape,
        emph=[1]{tipo,usa}, emphstyle={[1]\sffamily\bfseries},
        morekeywords={tint,forn,forsn},
        basewidth={0.47em,0.40em},
        columns=fixed, fontadjust, resetmargins, xrightmargin=5pt, xleftmargin=15pt,
        flexiblecolumns=false, tabsize=2, breaklines, breakatwhitespace=false, extendedchars=true,
        numbers=left, numberstyle=\tiny, stepnumber=1, numbersep=9pt,
        frame=l, framesep=3pt,
    }
   \csname lst@SetFirstLabel\endcsname}
  {\csname lst@SaveFirstLabel\endcsname}


%tipos basicos
\newcommand{\rea}{\ensuremath{\mathsf{Float}}}
\newcommand{\cha}{\ensuremath{\mathsf{Char}}}
\newcommand{\str}{\ensuremath{\mathsf{String}}}

\newcommand{\mcd}{\mathrm{mcd}}
\newcommand{\prm}[1]{\ensuremath{\mathsf{prm}(#1)}}
\newcommand{\sgd}[1]{\ensuremath{\mathsf{sgd}(#1)}}

\newcommand{\tuple}[2]{\ensuremath{#1 \times #2}}

%listas
\newcommand{\TLista}[1]{\ensuremath{seq \langle #1\rangle}}
\newcommand{\lvacia}{\ensuremath{[\ ]}}
\newcommand{\lv}{\ensuremath{[\ ]}}
\newcommand{\longitud}[1]{\ensuremath{|#1|}}
\newcommand{\cons}[1]{\ensuremath{\mathsf{addFirst}}(#1)}
\newcommand{\indice}[1]{\ensuremath{\mathsf{indice}}(#1)}
\newcommand{\conc}[1]{\ensuremath{\mathsf{concat}}(#1)}
\newcommand{\cab}[1]{\ensuremath{\mathsf{head}}(#1)}
\newcommand{\cola}[1]{\ensuremath{\mathsf{tail}}(#1)}
\newcommand{\sub}[1]{\ensuremath{\mathsf{subseq}}(#1)}
\newcommand{\en}[1]{\ensuremath{\mathsf{en}}(#1)}
\newcommand{\cuenta}[2]{\mathsf{cuenta}\ensuremath{(#1, #2)}}
\newcommand{\suma}[1]{\mathsf{suma}(#1)}
\newcommand{\twodots}{\ensuremath{\mathrm{..}}}
\newcommand{\masmas}{\ensuremath{++}}
\newcommand{\matriz}[1]{\TLista{\TLista{#1}}}

% Acumulador
\newcommand{\acum}[1]{\ensuremath{\mathsf{acum}}(#1)}
\newcommand{\acumselec}[3]{\ensuremath{\mathrm{acum}(#1 |  #2, #3)}}

% \selector{variable}{dominio}
\newcommand{\selector}[2]{#1~\ensuremath{\leftarrow}~#2}
\newcommand{\selec}{\ensuremath{\leftarrow}}

\newcommand{\pred}[3]{%
    {\normalfont\bfseries\ttfamily pred }%
    {\normalfont\ttfamily #1}%
    \ifthenelse{\equal{#2}{}}{}{\ (#2) }%
    \{\ensuremath{#3}\}%
    {\normalfont\bfseries\,\par}%
  }

\newenvironment{proc}[4][res]{%
  % El parametro 1 (opcional) es el nombre del resultado
  % El parametro 2 es el nombre del problema
  % El parametro 3 son los parametros
  % El parametro 4 es el tipo del resultado
  % Preambulo del ambiente problema
  % Tenemos que definir los comandos requiere, asegura, modifica y aux
  \newcommand{\pre}[2][]{%
    {\normalfont\bfseries\ttfamily Pre}%
    \ifthenelse{\equal{##1}{}}{}{\ {\normalfont\ttfamily ##1} :}\ %
    \{\ensuremath{##2}\}%
    {\normalfont\bfseries\,\par}%
  }
  \newcommand{\post}[2][]{%
    {\normalfont\bfseries\ttfamily Post}%
    \ifthenelse{\equal{##1}{}}{}{\ {\normalfont\ttfamily ##1} :}\
    \{\ensuremath{##2}\}%
    {\normalfont\bfseries\,\par}%
  }
  \renewcommand{\aux}[4]{%
    {\normalfont\bfseries\ttfamily fun\ }%
    {\normalfont\ttfamily ##1}%
    \ifthenelse{\equal{##2}{}}{}{\ (##2)}\ : ##3\, = \ensuremath{##4}%
    {\normalfont\bfseries\,;\par}%
  }
  \newcommand{\res}{#1}
  \vspace{1ex}
  \noindent
  \encabezadoDeProc{#1}{#2}{#3}{#4}
  % Abrimos la llave
  \{\par%
  \tocarEspacios
}
% Ahora viene el cierre del ambiente problema
{
  % Cerramos la llave
  \noindent\}
  \vspace{1ex}
}


  \newcommand{\aux}[4]{%
    {\normalfont\bfseries\ttfamily fun\ }%
    {\normalfont\ttfamily #1}%
    \ifthenelse{\equal{#2}{}}{}{\ (#2)}\ : #3\, = \ensuremath{#4}%
    {\normalfont\bfseries\,;\par}%
  }


% \newcommand{\pre}[1]{\textsf{pre}\ensuremath{(#1)}}

\newcommand{\procnom}[1]{\textsf{#1}}
\newcommand{\procil}[3]{\textsf{proc #1}\ensuremath{(#2) = #3}}
\newcommand{\procilsinres}[2]{\textsf{proc #1}\ensuremath{(#2)}}
\newcommand{\preil}[2]{\textsf{Pre #1: }\ensuremath{#2}}
\newcommand{\postil}[2]{\textsf{Post #1: }\ensuremath{#2}}
\newcommand{\auxil}[2]{\textsf{fun }\ensuremath{#1 = #2}}
\newcommand{\auxilc}[4]{\textsf{fun }\ensuremath{#1( #2 ): #3 = #4}}
\newcommand{\auxnom}[1]{\textsf{fun }\ensuremath{#1}}
\newcommand{\auxpred}[3]{\textsf{pred }\ensuremath{#1( #2 ) \{ #3 \}}}

\newcommand{\comentario}[1]{{/*\ #1\ */}}

\newcommand{\nom}[1]{\ensuremath{\mathsf{#1}}}


% En las practicas/parciales usamos numeros arabigos para los ejercicios.
% Aca cambiamos los enumerate comunes para que usen letras y numeros
% romanos
\newcommand{\arreglarincisos}{%
  \renewcommand{\theenumi}{\alph{enumi}}
  \renewcommand{\theenumii}{\roman{enumii}}
  \renewcommand{\labelenumi}{\theenumi)}
  \renewcommand{\labelenumii}{\theenumii)}
}



%%%%%%%%%%%%%%%%%%%%%%%%%%%%%% PARCIAL %%%%%%%%%%%%%%%%%%%%%%%%
\let\@xa\expandafter
\newcommand{\tituloparcial}{\centerline{\depto -- \lamateria}
  \centerline{\elnombre -- \lafecha}%
  \setlength{\TPHorizModule}{10mm} % Fija las unidades de textpos
  \setlength{\TPVertModule}{\TPHorizModule} % Fija las unidades de
                                % textpos
  \arreglarincisos
  \newcounter{total}% Este contador va a guardar cuantos incisos hay
                    % en el parcial. Si un ejercicio no tiene incisos,
                    % cuenta como un inciso.
  \newcounter{contgrilla} % Para hacer ciclos
  \newcounter{columnainicial} % Se van a usar para los cline cuando un
  \newcounter{columnafinal}   % ejercicio tenga incisos.
  \newcommand{\primerafila}{}
  \newcommand{\segundafila}{}
  \newcommand{\rayitas}{} % Esto va a guardar los \cline de los
                          % ejercicios con incisos, asi queda mas bonito
  \newcommand{\anchodegrilla}{20} % Es para textpos
  \newcommand{\izquierda}{7} % Estos dos le dicen a textpos donde colocar
  \newcommand{\abajo}{2}     % la grilla
  \newcommand{\anchodecasilla}{0.4cm}
  \setcounter{columnainicial}{1}
  \setcounter{total}{0}
  \newcounter{ejercicio}
  \setcounter{ejercicio}{0}
  \renewenvironment{ejercicio}[1]
  {%
    \stepcounter{ejercicio}\textbf{\noindent Ejercicio \theejercicio. [##1
      puntos]}% Formato
    \renewcommand\@currentlabel{\theejercicio}% Esto es para las
                                % referencias
    \newcommand{\invariante}[2]{%
      {\normalfont\bfseries\ttfamily invariante}%
      \ ####1\hspace{1em}####2%
    }%
    \newcommand{\Proc}[5][result]{
      \encabezadoDeProc{####1}{####2}{####3}{####4}\hspace{1em}####5}%
  }% Aca se termina el principio del ejercicio
  {% Ahora viene el final
    % Esto suma la cantidad de incisos o 1 si no hubo ninguno
    \ifthenelse{\equal{\value{enumi}}{0}}
    {\addtocounter{total}{1}}
    {\addtocounter{total}{\value{enumi}}}
    \ifthenelse{\equal{\value{ejercicio}}{1}}{}
    {
      \g@addto@macro\primerafila{&} % Si no estoy en el primer ej.
      \g@addto@macro\segundafila{&}
    }
    \ifthenelse{\equal{\value{enumi}}{0}}
    {% No tiene incisos
      \g@addto@macro\primerafila{\multicolumn{1}{|c|}}
      \bgroup% avoid overwriting somebody else's value of \tmp@a
      \protected@edef\tmp@a{\theejercicio}% expand as far as we can
      \@xa\g@addto@macro\@xa\primerafila\@xa{\tmp@a}%
      \egroup% restore old value of \tmp@a, effect of \g@addto.. is

      \stepcounter{columnainicial}
    }
    {% Tiene incisos
      % Primero ponemos el encabezado
      \g@addto@macro\primerafila{\multicolumn}% Ahora el numero de items
      \bgroup% avoid overwriting somebody else's value of \tmp@a
      \protected@edef\tmp@a{\arabic{enumi}}% expand as far as we can
      \@xa\g@addto@macro\@xa\primerafila\@xa{\tmp@a}%
      \egroup% restore old value of \tmp@a, effect of \g@addto.. is
      % global
      % Ahora el formato
      \g@addto@macro\primerafila{{|c|}}%
      % Ahora el numero de ejercicio
      \bgroup% avoid overwriting somebody else's value of \tmp@a
      \protected@edef\tmp@a{\theejercicio}% expand as far as we can
      \@xa\g@addto@macro\@xa\primerafila\@xa{\tmp@a}%
      \egroup% restore old value of \tmp@a, effect of \g@addto.. is
      % global
      % Ahora armamos la segunda fila
      \g@addto@macro\segundafila{\multicolumn{1}{|c|}{a}}%
      \setcounter{contgrilla}{1}
      \whiledo{\value{contgrilla}<\value{enumi}}
      {%
        \stepcounter{contgrilla}
        \g@addto@macro\segundafila{&\multicolumn{1}{|c|}}
        \bgroup% avoid overwriting somebody else's value of \tmp@a
        \protected@edef\tmp@a{\alph{contgrilla}}% expand as far as we can
        \@xa\g@addto@macro\@xa\segundafila\@xa{\tmp@a}%
        \egroup% restore old value of \tmp@a, effect of \g@addto.. is
        % global
      }
      % Ahora armo las rayitas
      \setcounter{columnafinal}{\value{columnainicial}}
      \addtocounter{columnafinal}{-1}
      \addtocounter{columnafinal}{\value{enumi}}
      \bgroup% avoid overwriting somebody else's value of \tmp@a
      \protected@edef\tmp@a{\noexpand\cline{%
          \thecolumnainicial-\thecolumnafinal}}%
      \@xa\g@addto@macro\@xa\rayitas\@xa{\tmp@a}%
      \egroup% restore old value of \tmp@a, effect of \g@addto.. is
      \setcounter{columnainicial}{\value{columnafinal}}
      \stepcounter{columnainicial}
    }
    \setcounter{enumi}{0}%
    \vspace{0.2cm}%
  }%
  \newcommand{\tercerafila}{}
  \newcommand{\armartercerafila}{
    \setcounter{contgrilla}{1}
    \whiledo{\value{contgrilla}<\value{total}}
    {\stepcounter{contgrilla}\g@addto@macro\tercerafila{&}}
  }
  \newcommand{\grilla}{%
    \g@addto@macro\primerafila{&\textbf{TOTAL}}
    \g@addto@macro\segundafila{&}
    \g@addto@macro\tercerafila{&}
    \armartercerafila
    \ifthenelse{\equal{\value{total}}{\value{ejercicio}}}
    {% No hubo incisos
      \begin{textblock}{\anchodegrilla}(\izquierda,\abajo)
        \begin{tabular}{|*{\value{total}}{p{\anchodecasilla}|}c|}
          \hline
          \primerafila\\
          \hline
          \tercerafila\\
          \tercerafila\\
          \hline
        \end{tabular}
      \end{textblock}
    }
    {% Hubo incisos
      \begin{textblock}{\anchodegrilla}(\izquierda,\abajo)
        \begin{tabular}{|*{\value{total}}{p{\anchodecasilla}|}c|}
          \hline
          \primerafila\\
          \rayitas
          \segundafila\\
          \hline
          \tercerafila\\
          \tercerafila\\
          \hline
        \end{tabular}
      \end{textblock}
    }
  }%
  \vspace{0.4cm}
  \textbf{Nro. de orden:}

  \textbf{LU:}

  \textbf{Apellidos:}

  \textbf{Nombres:}
  \vspace{0.5cm}
}



% AMBIENTE CONSIGNAS
% Se usa en el TP para ir agregando las cosas que tienen que resolver
% los alumnos.
% Dentro del ambiente hay que usar \item para cada consigna

\newcounter{consigna}
\setcounter{consigna}{0}

\newenvironment{consignas}{%
  \newcommand{\consigna}{\stepcounter{consigna}\textbf{\theconsigna.}}%
  \renewcommand{\ejercicio}[1]{\item ##1 }
  \renewcommand{\proc}[5][result]{\item
    \encabezadoDeProc{##1}{##2}{##3}{##4}\hspace{1em}##5}%
  \newcommand{\invariante}[2]{\item%
    {\normalfont\bfseries\ttfamily invariante}%
    \ ##1\hspace{1em}##2%
  }
  \renewcommand{\aux}[4]{\item%
    {\normalfont\bfseries\ttfamily aux\ }%
    {\normalfont\ttfamily ##1}%
    \ifthenelse{\equal{##2}{}}{}{\ (##2)}\ : ##3 \hspace{1em}##4%
  }
  % Comienza la lista de consignas
  \begin{list}{\consigna}{%
      \setlength{\itemsep}{0.5em}%
      \setlength{\parsep}{0cm}%
    }
}%
{\end{list}}



% para decidir si usar && o ^
\newcommand{\y}[0]{\ensuremath{\land}}

% macros de correctitud
\newcommand{\semanticComment}[2]{#1 \ensuremath{#2};}
\newcommand{\namedSemanticComment}[3]{#1 #2: \ensuremath{#3};}


\newcommand{\local}[1]{\semanticComment{local}{#1}}

\newcommand{\vale}[1]{\semanticComment{vale}{#1}}
\newcommand{\valeN}[2]{\namedSemanticComment{vale}{#1}{#2}}
\newcommand{\impl}[1]{\semanticComment{implica}{#1}}
\newcommand{\implN}[2]{\namedSemanticComment{implica}{#1}{#2}}
\newcommand{\estado}[1]{\semanticComment{estado}{#1}}

\newcommand{\invarianteCN}[2]{\namedSemanticComment{invariante}{#1}{#2}}
\newcommand{\invarianteC}[1]{\semanticComment{invariante}{#1}}
\newcommand{\varianteCN}[2]{\namedSemanticComment{variante}{#1}{#2}}
\newcommand{\varianteC}[1]{\semanticComment{variante}{#1}}

\usepackage{caratula}
\usepackage{scrextend}
\usepackage{listings}
\usepackage{color}
\usepackage{amsmath}

\definecolor{dkgreen}{rgb}{0,0.6,0}
\definecolor{gray}{rgb}{0.5,0.5,0.5}
\definecolor{mauve}{rgb}{0.58,0,0.82}

\lstset{frame=tb,
language=C++,
aboveskip=3mm,
belowskip=3mm,
showstringspaces=false,
columns=flexible,
basicstyle={\small\ttfamily},
numbers=none,
numberstyle=\tiny\color{gray},
keywordstyle=\color{blue},
commentstyle=\color{dkgreen},
stringstyle=\color{mauve},
breaklines=true,
breakatwhitespace=true,
tabsize=3
}


\begin{document}

%Carátula
\titulo{TP 1 - Reuniones Remotas}
\subtitulo{Grupo 3}
\fecha{22 de Mayo de 2020}
\materia{Algoritmos y Estructuras de Datos 1}
\integrante{González Narvarte, Francisco}{519/15}{francisco13\_95@live.com}
\integrante{Giménez, Iván Manuel}{374/18}{ivangimenez8727@gmail.com}
\integrante{Demare, Matías Nicolás}{762/19}{matiasdemare@gmail.com}
\integrante{Cubino, Santiago}{829/19}{sscubino@gmail.com}
\maketitle

%Creación de índice
\tableofcontents
\newpage

% End carátula

\newpage
\addcontentsline{toc}{section}{Demostraciones de complejidad}
\section*{Demostraciones de complejidad}
\vspace{0.5cm}
\subsection{acelerar}
\begin{lstlisting}
void acelerar(reunion& r, int prof, int freq) {
    for (int i = 0; i < r.size(); ++i) { // i)
        senial acelerado;
        senial original = get<0>(r[i]);  // O(n)
        for (int j = 1; j < get<0>(r[i]).size(); j=j+2) {  // ii)
            acelerado.push_back(original[j]);
        }
        get<0>(r[i]) = acelerado;  // O(n)
    }
}
\end{lstlisting}

i) En este ciclo se recorren todos los elementos de la reunión, o sea cada hablante, llamando m a la
longitud de la reunión nos termina costando $O(m)$.

ii) En este ciclo se recorre cada señal, y a pesar de que sólo se recorren las posiciones impares de la misma,
igual nos termina costando $O(n)$, donde n es el tamaño de cada señal. Esto se debe a que el cuerpo del ciclo es
$O(1)$, y se hacen $n/2$ iteraciones.

Como el resto son sólo operaciones elementales y asignaciones de señales (que es $O(n)$), la complejidad total nos queda $O(m \times n)$.


\newpage
\subsection{ordenar}
Calculo la complejidad en peor caso, llamando $m=|r|$ y $n=|(r[0])_0|$ \newline

\begin{lstlisting}
void ordenar(reunion& r, int freq, int prof) {
    for (int i = 1; i < r.size(); i++) {                       // guarda O(1), m-1 iteraciones
        burbujeoReuniones(r, i);                               // O(n*m)
    }                                                          // O(m^2*n) (guarda O(1), cuerpo O(n*m), m-1 iteraciones)
}                                                              // O(m^2*n)

void burbujeoReuniones (reunion &r, int i){
    for (int j=r.size()-1; j>i; j--){                          // guarda O(1), m-1 iteraciones
        if (tono(r[j].first) > tono(r[j-1].first)){            // O(n) + O(n) = O(n)
            swapReuniones(r, j, j-1);                          // O(n)
        }                                                      // O(n)
    }                                                          // O(n*m) (guarda O(1), cuerpo O(n), m-1 iteraciones)
}                                                              // O(n*m)

float tono(senial s){
    float t = 0;                                               // O(1)
    for (int i = 0; i < s.size(); i++) {                       // guarda O(1), n iteraciones
        t += abs(s[i]);                                        // O(1)
    }                                                          // O(n) (guarda y cuerpo O(1), n iteraciones)
    return t/s.size();                                         // O(1)
}                                                              // O(n)

void swapReuniones (reunion &r, int i, int j){
    pair<senial, hablante> a = r[i];                           // O(n) (se estan copiando seniales)
    pair<senial, hablante> b = r[j];                           // O(n)
    r[i] = b;                                                  // O(n)
    r[j] = a;                                                  // O(n)
}                                                              // O(n)
\end{lstlisting}

Como se puede ver, la complejidad en peor caso de la función ordenar es de $O(m^2\times n)$
Una cosa a notar es que en swapReuniones, todas las operaciones que se hacen implican copiar señales,
por lo que son todas $O(n)$


\newpage
\subsection{hablantesSuperpuestos}
En pocas palabras, la función hablantesSuperpuestos aplica la funcion silencios a las señales de cada hablante,
obteniendo asi, los indices de los intervalos de los silencios de cada hablante.
Al inicio, se declara un vector de ceros de tamaño n, que almacenara los instantes donde un hablante no estuvo en
silencio. \newline
Con los indices a y b se moverán entre el final de un intervalo de silencios y el comienzo de el siguiente. Y en cada
iteracion se escribira en el vector antes mencionado entre los indices a y b sin incluir, ``unos'', indicando de que un
hablante habló en ese periodo. \newline
Si en ningun momento se escribio un uno donde ya habia un uno, la funcion retornará falso, pues no se superponen. \newline

\begin{lstlisting}
bool hablantesSuperpuestos(reunion r, int prof, int freq, int umbral) {
    vector<int> registroDeNoSilencios(r[0].first.size(), 0);                               // O(n)

    for (int i = 0; i < r.size(); ++i) {                                       // guarda O(1), cuerpo O(n),m iteraciones
        vector<intervalo> silenciosDelHablante = silencios(r[i].first, prof, freq, umbral);// O(n), justificado abajo

        int a = 0;                                                                         // O(1)
        int b;                                                                             // O(1)
        for (int j = 0; j < silenciosDelHablante.size(); ++j) {   // guarda O(1), la cantidad de silencios del hablante
            b = silenciosDelHablante[j].first;                                             // O(1)
            for (int k = a; k < b; ++k) {                           // guarda O(1), la cantidad de iteraciones sera b-a
                if (registroDeNoSilencios[k] == 0) {                                       // O(1)
                    registroDeNoSilencios[k] = 1;                                          // O(1)
                } else {
                    return true;                                          // en el peor caso no se entrara a este branch
                }
            }
            a = silenciosDelHablante[j].second+1;                                          // O(1)
        }

        b = r[i].first.size();                                                             // Se repite una iteracion
        for (int k = a; k < b; ++k) {                                                      // mas el cuerpo del ciclo
            if (registroDeNoSilencios[k] == 0) {                                           // anterior.
                registroDeNoSilencios[k] = 1;
            } else {
                return true;
            }
        }
    }
    return false;
}
\end{lstlisting}

Podemos ver que el codigo cuenta con un ciclo que realiza m iteraciones con un cuerpo de complejidad $O(n)$, quedando una
complejidad de $O(n\times m)$. \newline
Veamos por que el cuerpo tiene complejidad n: \newline
Al principio de cada ciclo se llama a silencios, de complejidad $O(n)$, justificado más abajo. \newline
Los valores de a y b cambian en cada iteracion, pero tenemos la certeza de que la suma total de los $b-a$ de las
iteraciones estan acotados por $n$, pues basicamente lo que hacemos es escribir en el vector de hablantes, los instantes
donde el hablante estuvo hablando. Por lo tanto se entiende que la complejidad del siguiente ciclo es $O(n)$: \newline
for (int j = 0; j \textless silenciosDelHablante.size(); ++j) { \newline
    ... \newline
} \newline
El ultimo cuerpo fuera del ciclo también es $O(n)$ \newline
Y como adentro de ciclo grande tenemos suma de $O(n)$, el cuerpo es $O(n)$. \newline

\newpage
\addcontentsline{toc}{section}{Cálculo de complejidad}
\section*{Cálculo de complejidad}
\vspace{0.5cm}
\subsection{seEnojo}
Cuento la cantidad de operaciones en peor caso, llamando $n=|s|$

\begin{lstlisting}
bool seEnojo(senial s, int umbral, int prof, int freq) {
    bool resp = false;                                      // O(1)
    for (int i = 0; i < s.size(); i++) {                    // guarda O(1), n iteraciones
        for (int j = i+freq*2-1; j < s.size(); j++) {       // guarda O(1), (n-(i+freq*2-1)) iteraciones (si n<=(i+freq*2-1), entonces hace 0 iteraciones)
            resp = resp || tonoRango(s, i, j) > umbral;// O(n)
        }                                                   // O(n^2) si n>(i+freq*2-1). Si no, O(1). Como me interesa como se comporta en el limite, lo considero O(n)
    }                                                       // O(n^3) (cuerpo O(1), guarda O(1), n iteraciones)
    return resp;                                            // O(1)
}                                                           // O(n^3)
float tonoRango(senial s, int desde, int hasta){
    float t = 0;                                            // O(1)
    for (int i = desde; i <= hasta; i++) {                  // guarda O(1), n iteraciones
        t += abs(s[i]);                                     // O(1)
    }                                                       // O(n) (guarda y cuerpo O(1), n iteraciones)
    return t/s.size();                                      // O(1)
}                                                           // O(n)
\end{lstlisting}

Como se puede ver, seEnojo consiste de 3 ciclos for anidados, cada uno se ejecuta en peor caso $O(n)$ veces.
Considero que $n>(i+freq*2-1)$, ya que freq es constante, por lo que siempre va a existir un $n$ tal que esto se cumpla,
y para la notación O grande, me interesa como se comporta la complejidad en el limite.
Esto resulta en una complejidad $O(n^3)$ para la función seEnojo

\newpage
\subsection{silencios}

\begin{lstlisting}
vector<intervalo> silencios(senial s, int prof, int freq, int umbral) {
    // Nota: para el tiemo de ejecucion de peor caso estoy considerando el caso
    // en el que toda la senial es silencio, leer aclaracion abajo
    vector<intervalo> intervalos;                                                       // O(1)
    for (int i = 0; i < s.size(); i++) {                                                // guarda O(1), 1 iteracion
        if(abs(s[i]) < umbral){                                                         // O(1)
            int avanzarHasta = i;                                                       // O(1)
            for (int j = i; j < s.size(); j++) {                                        // guarda O(1), n iteraciones
                avanzarHasta = j+1;                                                     // O(1)
                if(abs(s[j]) < umbral && (j == s.size()-1 || abs(s[j+1]) >= umbral)){       // O(1)
                    intervalo interv = make_pair(i, j);                                 // O(1)
                    if(duracion(interv, freq) >= 0.2)                                   // O(1) + O(1)
                        intervalos.push_back(interv);                                   // O(1)
                    break;                                                              // O(1)
                }
            }                                                                           // O(n) (guarda y cuerpo O(1), n iteraciones)
            i = avanzarHasta;                                                           // O(1)
        }
    }                                                                                   // O(n) (guarda O(1), cuerpo O(n), 1 iteracion)
    return intervalos;                                                                  // O(1)
}                                                                                       // O(n)

float duracion (intervalo interv, int freq){
    return (interv.second + 1 - interv.first) * 1.f / freq;                             // O(1)
}
\end{lstlisting}

Notar que sólo se recorre cada posición de s una vez, ya que si entra al for de adentro, despúes actualizo i
al valor de j+1, ya que sé que o no hay silencios en [i,j] o que ya fueron agregados a intervalos.

Dado que todas las operaciones dentro de ambos fors son $O(1)$, y se ejecutan como máximo $n=|s|$ veces, puedo
asegurar que el algoritmo es $O(n)$.

Además, el tiempo de peor caso es el que se obtendrá si s es todo silencio, ya que forzará a que se ejecute n
veces la guarda $if(abs(s[j]) < umbral \&\& (j == s.size()-1 || abs(s[j+1]) >= umbral))$ y el incremento de
la variable avanzarHasta, que tienen un tiempo de ejecución mayor a lo que ocurre cuando no hay silencios,
que hace que se ejecute únicamente la guarda $if(abs(s[i]) < umbral)$. Entonces, lo comentado en el código el tiempo de
ejecución en peor caso, considerando que éste se da al ser todo silencios.


\newpage
\subsection{filtradoMediana}
\begin{lstlisting}
void filtradoMediana(senial& s, int R, int prof, int freq){
    senial aux = s;
    int i = R;
    while (0 <= i-R && i+R+1 < aux.size()) {  // i)
        senial w;
        for (int j = i-R; j < i+R+1; ++j) {  // ii)
            w.push_back(aux[j]);
        }
        w = bubbleSort(w);  // iii)
        s[i] = w[R];
        i++;
    }
}
\end{lstlisting}

i) En el peor caso en este ciclo se recorre la señal original entera por lo que su complejidad es $O(m)$ con m siendo el tamaño del vector original.

ii) En el peor caso, con R = 4, se recorren a lo sumo 9 posiciones (2*R+1), por lo que no suma al total de la complejidad.

iii) Como vimos en la teórica, la complejidad del algoritmo bubbleSort es $O(n^2)$ donde n es la longitud de w y se aplica sobre todo w, como el tamaño de w es a lo sumo 9, en el peor caso termina costando $O(9^2)$, por lo que tampoco suma al total de la complejidad.

Lo definido antes del primer ciclo son operaciones elementales que sólo cuestan $O(1)$, la declaración de la señal y el pushback también y por último las últimas 2 asignaciones también, por lo que no suman al total de la complejidad.

Luego, la complejidad total es $O(m)$.

\end{document}