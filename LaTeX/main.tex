\documentclass{article}
\input{Algo1Macros}
\usepackage{caratula}
\usepackage{scrextend}
\usepackage{listings}
\usepackage{color}
\usepackage{amsmath}

\definecolor{dkgreen}{rgb}{0,0.6,0}
\definecolor{gray}{rgb}{0.5,0.5,0.5}
\definecolor{mauve}{rgb}{0.58,0,0.82}

\lstset{frame=tb,
language=C++,
aboveskip=3mm,
belowskip=3mm,
showstringspaces=false,
columns=flexible,
basicstyle={\small\ttfamily},
numbers=none,
numberstyle=\tiny\color{gray},
keywordstyle=\color{blue},
commentstyle=\color{dkgreen},
stringstyle=\color{mauve},
breaklines=true,
breakatwhitespace=true,
tabsize=3
}


\begin{document}

%Carátula
\titulo{TP 1 - Reuniones Remotas}
\subtitulo{Grupo 3}
\fecha{22 de Mayo de 2020}
\materia{Algoritmos y Estructuras de Datos 1}
\integrante{González Narvarte, Francisco}{519/15}{francisco13\_95@live.com}
\integrante{Giménez, Iván Manuel}{374/18}{ivangimenez8727@gmail.com}
\integrante{Demare, Matías Nicolás}{762/19}{matiasdemare@gmail.com}
\integrante{Cubino, Santiago}{829/19}{sscubino@gmail.com}
\maketitle

%Creación de índice
\tableofcontents
\newpage

% End carátula


\addcontentsline{toc}{section}{Notas}
\section*{Notas}
\vspace{0.5cm}


\newpage
\addcontentsline{toc}{section}{Demostraciones de complejidad}
\section*{Demostraciones de complejidad}
\vspace{0.5cm}
\subsection{acelerar}
\begin{lstlisting}
void acelerar(reunion& r, int prof, int freq) {
    for (int i = 0; i < r.size(); ++i) { // i)
        senial acelerado;
        senial original = get<0>(r[i]);
        for (int j = 1; j < get<0>(r[i]).size(); j=j+2) {  // ii)
            acelerado.push_back(original[j]);
        }
        get<0>(r[i]) = acelerado;
    }
    return; 
}
\end{lstlisting}
i) En este ciclo se recorren todos los elementos de la reunión, o sea cada hablante, llamando m a la longitud de la reunión nos termina costando $O(m)$.

ii) Mientras que en este ciclo se recorre cada señal, a pesar de que sólo se recorren las posiciones impares igual nos termina costando $O(n)$, donde n es el tamaño de cada señal.

Como el resto de las sólo son operaciones elementales la complejidad total nos queda $O(m \times n)$.


\newpage
\subsection{ordenar}
Cuento la cantidad de operaciones en peor caso, llamando $m=|r|$ y $n=|(r[0])_0|$
Cuento como operaciones elementales: \newline
(i) Los accesos a valores de las variables. \newline
(ii) Los accesos a elementos de vectores. \newline
(iii) Los accesos a elementos de tuplas (operación $first$). \newline
(iv) Las inicializaciones de las variables. \newline
(v) Las asignaciones. \newline
(vi) Las operaciones matemáticas o lógicas. \newline
(vii) La operación $size$. \newline
(viii) Los incrementos del tipo ”$i++$”, y los decrementos del tipo ”$j−−$”. \newline
(ix) La operación $return$. \newline

\begin{lstlisting}
void ordenar(reunion& r, int freq, int prof) {
    for (int i = 1; i < r.size(); i++) {                       // 5, m-1 iteraciones
        burbujeo(r, i);                                        // 2 + 49m - 49 + 22n*m - 22n = 49m - 47 + 22 n*m - 22n
    }                                                          // t(n,m) = 49m^2-96m + 22n*m^2 - 44n*m + 47 + 22n -> O(n*m^2)
}

void burbujeoReuniones (reunion &r, int i){
    for (int j=r.size()-1; j>i; j--){                          // 5, m-1 iteraciones
        if (tono(r[j].first) > tono(r[j-1].first)){            // 9 + 2 * (9+11n) = 27 + 22n
            cambiar(r, j, j-1);                                // 14 + 3 = 17
        }                                                      // t(n,m) = 5 + (5+17+27+22n) * (m-1) = 49m - 49 + 22n*m - 22n
    }                                                          // t(n,m) = 49m - 49 + 22n*m - 22n -> O(n*m)
}
float tono(senial s){
    float t = 0;                                               // 1
    for (int i = 0; i < s.size(); i++) {                       // 5, n iteraciones
        t += abs(s[i]);                                        // 6
    }                                                          // t(n) = 5 + 11n
    return t/s.size();                                         // 3
}                                                              // t(n) = 9 + 11n -> O(n)
void swapReuniones (reunion &r, int i, int j){
    pair<senial, hablante> a = r[i];                           // 3
    pair<senial, hablante> b = r[j];                           // 3
    r[i] = b;                                                  // 4
    r[j] = a;                                                  // 4

}                                                              // t = 14 -> O(1)
\end{lstlisting}


\newpage
\subsection{hablantesSuperpuestos}
En pocas palabras, la función hablantesSuperpuestos aplica la funcion silencios a las señales de cada hablante,
obteniendo asi, los indices de los intervalos de los silencios de cada hablante.
Al inicio, se declara un vector de ceros de tamaño n, que almasenara los instantes donde un hablante no estuvo en
silencio. \newline
Con los indices a y b se moverán entre el final de un intervalo de silencios y el comienzo de el siguiente. Y en cada
iteracion se escribira en el vector antes mensionado entre los indices a y b sin incluir, "unos", indicando de que un
hablante habló en ese periodo. \newline
Si en ningun momento se escribio un uno donde ya habia un uno, la funcion retornará falso, pues no se superponen. \newline

\begin{lstlisting}
bool hablantesSuperpuestos(reunion r, int prof, int freq, int umbral) {
    vector<int> registroDeNoSilencios(r[0].first.size(), 0);                               // O(n)

    for (int i = 0; i < r.size(); ++i) {                                       // guarda O(1), cuerpo O(n),m iteraciones
        vector<intervalo> silenciosDelHablante = silencios(r[i].first, prof, freq, umbral);// O(n), justificado abajo

        int a = 0;                                                                         // O(1)
        int b;                                                                             // O(1)
        for (int j = 0; j < silenciosDelHablante.size(); ++j) {   // guarda O(1), la cantidad de silencios del hablante
            b = silenciosDelHablante[j].first;                                             // O(1)
            for (int k = a; k < b; ++k) {                           // guarda O(1), la cantidad de iteraciones sera b-a
                if (registroDeNoSilencios[k] == 0) {                                       // O(1)
                    registroDeNoSilencios[k] = 1;                                          // O(1)
                } else {
                    return true;                                          // en el peor caso no se entrara a este branch
                }
            }
            a = silenciosDelHablante[j].second+1;                                          // O(1)
        }

        b = r[i].first.size();                                                             // Se repite una iteracion
        for (int k = a; k < b; ++k) {                                                      // mas el cuerpo del ciclo
            if (registroDeNoSilencios[k] == 0) {                                           // anterior.
                registroDeNoSilencios[k] = 1;
            } else {
                return true;
            }
        }
    }
    return false;
}
\end{lstlisting}

Podemos ver que el codigo cuenta con un ciclo que realiza m iteraciones con un cuerpo de complejidad O(n), quedando una
complejidad de O(n*m). \newline
Veamos por que el cuerpo tiene complejidad n: \newline
Al principio de cada ciclo se llama a ordenar, de complejidad O(n), justificado más abajo. \newline
Los valores de a y b cambian en cada iteracion, pero tenemos la certesa de que la suma total de los b-a de las
iteraciones estan acotados por n, pues basicamente lo que hacemos es escribir en el vector de hablantes, los instantes
donde el hablante estuvo hablando. Por lo tanto se entiende que la complejidad del siguiente ciclo es O(n): \newline
for (int j = 0; j < silenciosDelHablante.size(); ++j) { \newline
    ... \newline
} \newline
El ultimo cuerpo fuera del ciclo tamb es O(n) \newline
Y como adentro de ciclo grande tenemos suma de O(n), el cuerpo es O(n). \newline

\newpage
\addcontentsline{toc}{section}{Cálculo de complejidad}
\section*{Cálculo de complejidad}
\vspace{0.5cm}
\subsection{seEnojo}
Cuento la cantidad de operaciones en peor caso, llamando $n=|s|$

\begin{lstlisting}
bool seEnojo(senial s, int umbral, int prof, int freq) {
    bool resp = false;                                      // O(1)
    for (int i = 0; i < s.size(); i++) {                    // guarda O(1), n iteraciones
        for (int j = i+freq*2-1; j < s.size(); j++) {       // guarda O(1), (n-(i+freq*2-1)) iteraciones (si n<=(i+freq*2-1), entonces hace 0 iteraciones)
            resp = resp || tonoRango(s, i, j) > umbral;// O(n)
        }                                                   // O(n^2) si n>(i+freq*2-1). Si no, O(1). Como me interesa como se comporta en el limite, lo considero O(n)
    }                                                       // O(n^3) (cuerpo O(1), guarda O(1), n iteraciones)
    return resp;                                            // O(1)
}                                                           // O(n^3)
float tonoRango(senial s, int desde, int hasta){
    float t = 0;                                            // O(1)
    for (int i = desde; i <= hasta; i++) {                  // guarda O(1), n iteraciones
        t += abs(s[i]);                                     // O(1)
    }                                                       // O(n) (guarda y cuerpo O(1), n iteraciones)
    return t/s.size();                                      // O(1)
}                                                           // O(n)
\end{lstlisting}

Como se puede ver, seEnojo consiste de 3 ciclos for anidados, cada uno se ejecuta en peor caso O(n) veces.
Considero que $n>(i+freq*2-1)$, ya que freq es constante, por lo que siempre va a existir un n tal que esto se cumpla,
y para la notación O grande, me interesa como se comporta la complejidad en el limite.
Esto resulta en una complejidad $O(n^3)$ para la función seEnojo

\newpage
\subsection{silencios}

\begin{lstlisting}
vector<intervalo> silencios(senial s, int prof, int freq, int umbral) {
    // Nota: para el tiemo de ejecucion de peor caso estoy considerando el caso
    // en el que toda la senial es silencio, leer aclaracion abajo
    vector<intervalo> intervalos;                                                       // O(1)
    for (int i = 0; i < s.size(); i++) {                                                // guarda O(1), 1 iteracion
        if(abs(s[i]) < umbral){                                                         // O(1)
            int avanzarHasta = i;                                                       // O(1)
            for (int j = i; j < s.size(); j++) {                                        // guarda O(1), n iteraciones
                avanzarHasta = j+1;                                                     // O(1)
                if(abs(s[j]) < umbral && (j == s.size()-1 || abs(s[j+1]) >= umbral)){       // O(1)
                    intervalo interv = make_pair(i, j);                                 // O(1)
                    if(duracion(interv, freq) >= 0.2)                                   // O(1) + O(1)
                        intervalos.push_back(interv);                                   // O(1)
                    break;                                                              // O(1)
                }
            }                                                                           // O(n) (guarda y cuerpo O(1), n iteraciones)
            i = avanzarHasta;                                                           // O(1)
        }
    }                                                                                   // O(n) (guarda O(1), cuerpo O(n), 1 iteracion)
    return intervalos;                                                                  // O(1)
}                                                                                       // O(n)

float duracion (intervalo interv, int freq){
    return (interv.second + 1 - interv.first) * 1.f / freq;                             // O(1)
}
\end{lstlisting}

Notar que sólo se recorre cada posición de s una vez, ya que si entra al for de adentro, despúes actualizo i
al valor de j+1, ya que sé que o no hay silencios en [i,j] o que ya fueron agregados a intervalos.

Dado que todas las operaciones dentro de ambos fors son O(1), y se ejecutan como máximo $n=|s|$ veces, puedo
asegurar que el algoritmo es O(n).

Además, el tiempo de peor caso es el que se obtendrá si s es todo silencio, ya que forzará a que se ejecute n
veces la guarda $if(abs(s[j]) < umbral \&\& (j == s.size()-1 || abs(s[j+1]) >= umbral))$ y el incremento de
la variable avanzarHasta, que tienen un tiempo de ejecución mayor a lo que ocurre cuando no hay silencios,
que hace que se ejecute únicamente la guarda $if(abs(s[i]) < umbral)$. Entonces, lo comentado en el código el tiempo de
ejecución en peor caso, considerando que éste se da al ser todo silencios.


\newpage
\subsection{filtradoMediana}
\begin{lstlisting}
void filtradoMediana(senial& s, int R, int prof, int freq){
    senial aux = s;
    int i = R;
    while (0 <= i-R && i+R+1 < aux.size()) {  // i)
        senial w;
        for (int j = i-R; j < i+R+1; ++j) {  // ii)
            w.push_back(aux[j]);
        }
        w = bubbleSort(w);  // iii)
        s[i] = w[R];
        i++;
    }
    return;
}
\end{lstlisting}

i) En el peor caso en este ciclo se recorre la señal original entera por lo que su complejidad es $O(m)$ con m siendo el tamaño del vector original.

ii) Mientras que en este ciclo pasa lo mismo excepto con el último elemento de la señal original, esto no cambia que su complejidad también sea $O(m)$.

iii) Como vimos en la teórica, la complejidad del algoritmo bubbleSort es $O(n^2)$ donde n es la longitud de w y se aplica sobre todo w.

Lo definido antes del primer ciclo son operaciones elementales que sólo cuestan $O(1)$, la declaración de la señal y el pushback también y por último las últimas 2 asignaciones también, por lo que no suman al total de la complejidad.

Luego, la complejidad total es $O(m^2 \times n^2)$.

\end{document}